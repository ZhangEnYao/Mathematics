\Notion{
    Summary
}{
    Binary Relation Structures: Domain, Range, Field, Image.\\
    Binary Relation Operators: Inverse, Composition, Cartesian Product,\\
    Binary Relation Instances: Membership Relation, Identity Relation.\\
    Binary Relation Generalizations: Unary Relation, Ternary Relation.\\
}

\Definition{
    Binary Relation
}{
    A set $A$ is a binary relation if all elements of $R$ are ordered pairs,
    i.e., if for any $z$ $\in$ $R$ there exist $x$ and $y$ such that $z$ $=$ $(x, y)$.
    A binary relation $A$ is in $A$ if and only if $R$ $\subseteq$ $A^{2}$ 
}
\Note{
    Relations between objects of two sorts called binary relations.
}
\Note{
    A binary relation is determined by specifying all ordered pairs of objects in that relation;
    it does not matter by what property the set of these ordered pairs is described.
}

\Definition{
    Domain
}{
    Let $A$ be a binary relation.
    The set of all $x$ which are in relation $A$ with some $y$ is called the domain of $R$ and denoted by $\DomainFunction{dom}{R}$.
}
\Note{
    $\DomainFunction{dom}{R}$ is the set of all first coordinates of ordered pairs in $R$.
}
\Definition{
    Range
}{
    Let $A$ be a binary relation.
    The set of all $y$ such that, for some $x$, $x$ is in relation $R$ with $y$ is called the range of $A$, denoted by $\DomainFunction{ran}{R}$.
}
\Note{
    $\DomainFunction{ran}{R}$ is the set of all second coordinates of ordered pairs in $R$.
}
\Definition{
    Field
}{
    Let $A$ be a binary relation.
    The set $\DomainFunction{dom}{R}$ $\cup$ $\DomainFunction{ran}{R}$ is called the field of $R$ and is denoted by $\DomainFunction{field}{R}$.
    If $\DomainFunction{field}{R}$ $\subseteq$ $X$, we say that $R$ is a relation in $X$ or that $R$ is a relation between elements of $X$.
}
\Proposition{}{
    Both \Keywords{$\DomainFunction{dom}{R}$} and \Keywords{$\DomainFunction{ran}{R}$} \Keywords{exist} for any relation $R$.
}{
    Obviously.
}

\Definition{
    Image
}{
    The image of $A$ under $R$ is the set of all $y$ from the range of $R$ related in $R$ to some element of $A$; it is denoted by $R[A]$.
}
\Definition{
    Inverse Image
}{
    The inverse image of $B$ under $R$ is the set of all $x$ from the domain of $R$ related in $R$ to some element of $B$; it is denoted by $\Inverse{R}[A]$.
}
\Proposition{}{
    \Keywords{$\DomainFunction{dom}{R}$} $=$ \Keywords{$\DomainFunction{ran}{\Inverse{R}}$}.
}{
    Obviously
}
\Proposition{}{
    The \Keywords{inverse image of $B$ under $R$} is equal to the \Keywords{image of $B$ under $\Inverse{R}$}.
}{
    Obviously
}

\Definition{
    Inverse
}{
    Let $R$ be a binary relation.
    The inverse of $R$ is the set $$\Inverse{R}=\SetComprehension{z}{z=(x, y) \land  \QuantifierCondition{\exists}{x, y}{(y, x) \in R}}.$$
}
\Definition{
    Composition
}{
    Let $R$ and $S$ be binary relations.
    The composition of $R$ and $S$ is the relation.
    $$ \Compose{S}{R} = \SetComprehension{(x, z)}{\QuantifierCondition{\exists}{y}{(x, y) \in R \land (y, z) \in S}}.$$
}
\Note{
    $(x, z)$ $\in$ $\Compose{S}{R}$ means that for some $y$, $(x, y) \in R$ and $(y, z) \in S$.
}

\Definition{
    Cartesian Product
}{
    Let $A$ and $B$ be sets.
    The set of all ordered pairs whose first coordinate is from $A$ and whose second coordinate is from $B$ is called the cartesian product of $A$ and $B$ and denoted $\CartesianProduct{A}{B}$.
}
\Note{
    $\CartesianProduct{A}{B}$ is a relation in which every element of $A$ is related to every element of $B$.
}
\Proposition{}{
    \Keywords{$\CartesianProduct{A}{B}$ exists}.
}{
    $\CartesianProduct{A}{B} = \SetComprehension{w \in \Powerset{\Powerset{\Set{a, b}}}}{w = (a, b) \land \QuantifierCondition{\exists}{a, b}{a \in A \land b \in B}}.$
}
\Proposition{}{
    $\CartesianProduct{(\CartesianProduct{A}{B})}{C}$ $=$ $\CartesianProduct{\CartesianProduct{A}{B}}{C}$.
}{
    Obviously.
}
\Note{
    $\CartesianProduct{\CartesianProduct{A}{B}}{C} = \SetComprehension{(a, b, c)}{a \in A \land b \in B \land c \in C}.$
}

\Definition{
    Unary Relation
}{
    A unary relation is any set.
    A unary relation in $A$ is any subset of $A$.
}

\Definition{
    Ternary Relation
}{
    A ternary relation is a set of unordered triples.
    More explicitly, $S$ is a ternary relation if for every $u$ $\in$ $S$, there exist $x$, $y$, and $z$ such that $u = (x, y, z).$
    If $S$ $\subseteq$ $A^{3}$, we say that $S$ is a ternary relation in $A$.
}

\Definition{
    Membership Relation
}{
    The membership relation on $A$ is defined by $${\in}_{A} = \SetComprehension{(a, b)}{a, b \in A \land a \in b}.$$
}
\Definition{
    Identity Relation
}{
    The identity relation on $A$ is defined by $${Id}_{A} = \SetComprehension{(a, b)}{a, b \in A \land a = b}.$$
}
