\Notion{
    Summary
}{
    Undering.
}

\Definition{
    Binary Relation
}{
    A set $A$ is a binary relation if all elements of $R$ are ordered pairs,
    i.e., if for any $z$ $\in$ $R$ there exist $x$ and $y$ such that $z$ $=$ $(x, y)$.
}
\Note{
    Relations between objects of two sorts called binary relations.
}
\Note{
    A binary relation is determined by specifying all ordered pairs of objects in that relation;
    it does not matter by what property the set of these ordered pairs is described.
}

\Definition{}{
    Let $A$ be a binary relation.
    \List{enumerate}{
        \item{
            The set of all $x$ which are in relation $A$ with some $y$ is called the domain of $R$ and denoted by \Notation{dom}{$R$}.
        }
        \item{
            The set of all $y$ such that, for some $x$, $x$ is in relation $R$ with $y$ is called the range of $A$, denoted by \Notation{ran}{$R$}.
        }
        \item{
            The set \Notation{dom}{$R$} $\cup$ \Notation{ran}{$R$} is called the field of $R$ and is denoted by \Notation{field}{$R$}.
        }
        \item{
            If \Notation{field}{$R$} $\subseteq$ $X$, we say that $R$ is a relation in $X$ or that $R$ is a relation between elements of $X$.
        }
    }
}
\Note{
    \Notation{dom}{$R$} is the set of all first coordinates of ordered pairs in $R$.
}
\Note{
    \Notation{ran}{$R$} is the set of all second coordinates of ordered pairs in $R$.
}

\Proposition{}{
    Both \Notation{dom}{$R$} and \Notation{ran}{$R$} exist for any relation $R$.
}{
    Obviously.
}