\Notion{
    Summary
}{
    Binary Relation Structures: Domain, Range, Field, Image.\\
    Binary Relation Operators: Inverse, Composition, Cartesian Product.\\
    Binary Relation Instances: Membership Relation, Identity Relation.\\
    Binary Relation Generalizations: Unary Relation, Ternary Relation.\\
}

\Definition{
    Binary Relation
}{
    A set $A$ is a binary relation if all elements of $R$ are ordered pairs,
    i.e., if for any $z$ $\in$ $R$ there exist $x$ and $y$ such that $z$ $=$ $(x, y)$.
    A binary relation $A$ is in $A$ if and only if $R$ $\subseteq$ $A^{2}$ 
}
\Note{
    Relations between objects of two sorts called binary relations.
}
\Note{
    A binary relation is determined by specifying all ordered pairs of objects in that relation;
    it does not matter by what property the set of these ordered pairs is described.
}

\Definition{
    Domain
}{
    Let $A$ be a binary relation.
    The set of all $x$ which are in relation $A$ with some $y$ is called the domain of $R$ and denoted by $\domain{R}$.
}
\Note{
    $\domain{R}$ is the set of all first coordinates of ordered pairs in $R$.
}
\Definition{
    Range
}{
    Let $A$ be a binary relation.
    The set of all $y$ such that, for some $x$, $x$ is in relation $R$ with $y$ is called the range of $A$, denoted by $\range{R}$.
}
\Note{
    $\range{R}$ is the set of all second coordinates of ordered pairs in $R$.
}
\Definition{
    Field
}{
    Let $A$ be a binary relation.
    The set $\domain{R}$ $\cup$ $\range{R}$ is called the field of $R$ and is denoted by $\field{R}$.
    If $\field{R}$ $\subseteq$ $X$, we say that $R$ is a relation in $X$ or that $R$ is a relation between elements of $X$.
}
\Proposition{}{
    Both \Keywords{$\domain{R}$} and \Keywords{$\range{R}$} \Keywords{exist} for any relation $R$.
}{
    Obviously.
}

\Definition{
    Image
}{
    The image of $A$ under $R$ is the set of all $y$ from the range of $R$ related in $R$ to some element of $A$; it is denoted by $R[A]$.
}
\Definition{
    Inverse Image
}{
    The inverse image of $B$ under $R$ is the set of all $x$ from the domain of $R$ related in $R$ to some element of $B$; it is denoted by $\Inverse{R}[A]$.
}
\Proposition{}{
    \Keywords{$\domain{R}$} $=$ \Keywords{$\range{\Inverse{R}}$}.
}{
    Obviously
}
\Proposition{}{
    The \Keywords{inverse image of $B$ under $R$} is equal to the \Keywords{image of $B$ under $\Inverse{R}$}.
}{
    Obviously
}

\Definition{
    Inverse
}{
    Let $R$ be a binary relation.
    The inverse of $R$ is the set $$\Inverse{R}=\SetComprehension{z}{z=(x, y) \land  \QuantifierCondition{\exists}{x, y}{(y, x) \in R}}.$$
}
\Definition{
    Composition
}{
    Let $R$ and $S$ be binary relations.
    The composition of $R$ and $S$ is the relation.
    $$ S \composition R = \SetComprehension{(x, z)}{\QuantifierCondition{\exists}{y}{(x, y) \in R \land (y, z) \in S}}.$$
}
\Note{
    $(x, z)$ $\in$ $S \composition R$ means that for some $y$, $(x, y) \in R$ and $(y, z) \in S$.
}

\Definition{
    Cartesian Product
}{
    Let $A$ and $B$ be sets.
    The set of all ordered pairs whose first coordinate is from $A$ and whose second coordinate is from $B$ is called the cartesian product of $A$ and $B$ and denoted $A \cartesianproduct B$.
}
\Note{
    $A \cartesianproduct B$ is a relation in which every element of $A$ is related to every element of $B$.
}
\Proposition{}{
    \Keywords{$A \cartesianproduct B$ exists}.
}{
    $A \cartesianproduct B = \SetComprehension{w \in \Powerset{\Powerset{\Set{a, b}}}}{w = (a, b) \land \QuantifierCondition{\exists}{a, b}{a \in A \land b \in B}}.$
}
\Proposition{}{
    $(A \cartesianproduct B) \cartesianproduct C$ $=$ $A \cartesianproduct B \cartesianproduct C$.
}{
    Obviously.
}
\Note{
    $A \cartesianproduct B \cartesianproduct C = \SetComprehension{(a, b, c)}{a \in A \land b \in B \land c \in C}.$
}

\Definition{
    Unary Relation
}{
    A unary relation is any set.
    A unary relation in $A$ is any subset of $A$.
}

\Definition{
    Ternary Relation
}{
    A ternary relation is a set of unordered triples.
    More explicitly, $S$ is a ternary relation if for every $u$ $\in$ $S$, there exist $x$, $y$, and $z$ such that $u = (x, y, z).$
    If $S$ $\subseteq$ $A^{3}$, we say that $S$ is a ternary relation in $A$.
}

\Definition{
    Membership Relation
}{
    The membership relation on $A$ is defined by $${\in}_{A} = \SetComprehension{(a, b)}{a, b \in A \land a \in b}.$$
}
\Definition{
    Identity Relation
}{
    The identity relation on $A$ is defined by $${\identityrelation}_{A} = \SetComprehension{(a, b)}{a, b \in A \land a = b}.$$
}

\Property{}{
    Let $A$ be a binary relation.
    \Keywords{$\domain{R}$} and \Keywords{$\range{R}$} \Keywords{exist}.
}{
    $(x, y) \in R$ implies $x, y \in \bigcup(\bigcup R)$.
}
\Remark{
    Union flatten a set. Powerset structure a set.
}
\Property{}{
    \Keywords{$\Inverse{R}$} \Keywords{exist}.
}{
    $\Inverse{R}$ $\subseteq$ $\domain{R} \cartesianproduct \range{R}$.
}
\Property{}{
    \Keywords{$S \composition R$} \Keywords{exist}.
}{
    $S \composition R$ $\subseteq$ $\domain{R} \cartesianproduct \range{S}$.
}
\Property{}{
    For any three binary relations $R$, $S$, and $T$,
    \Keywords{$T \composition (S \composition R)$ $=$ $(T \composition S) \composition R$}.
}{
    $S \composition R$ $\subseteq$ $\domain{R} \cartesianproduct \range{S}$.
}
\Property{}{
    \Keywords{$A \cartesianproduct B \cartesianproduct C$} \Keywords{exist}.
}{
    Obviously.
}

\Property{}{
    Let $R$ be a binary relation and $A$ and $B$ sets.
    \List{enumerate}{
        \item{
            \Keywords{$R[A \cup B]$ $=$ $R[A] \cup R[B]$}
        }
        \item{
            \Keywords{$R[A \cap B]$ $\subseteq$ $R[A] \cap R[B]$}
        }
        \item{
            \Keywords{$R[A - B]$ $\supseteq$ $R[A] - R[B]$}
        }
        \item{
            \Keywords{$\Inverse{R}[A \cup B]$ $=$ $\Inverse{R}[A] \cup \Inverse{R}[B]$}
        }
        \item{
            \Keywords{$\Inverse{R}[A \cap B]$ $\subseteq$ $\Inverse{R}[A] \cap \Inverse{R}[B]$}
        }
        \item{
            \Keywords{$\Inverse{R}[A - B]$ $\supseteq$ $\Inverse{R}[A] - \Inverse{R}[B]$}
        }
        \item{
            \Keywords{$\Inverse{R}[A - B]$ $\supseteq$ $\Inverse{R}[A] - \Inverse{R}[B]$}
        }
        \item{
            \Keywords{$\Inverse{R}[R[A]]$ $\supseteq$ $A \cap \domain{R}$}
        }
        \item{
            \Keywords{$R[\Inverse{R}[B]]$ $\supseteq$ $B \cap \range{R}$}
        }
    }
}{
    Obviously.
}

\Property{}{
    Let $R$ $\subseteq$ $X \cartesianproduct Y$.
    \List{enumerate}{
        \item{
            \Keywords{$R[X]$ $=$ $\range{R}$}.
        }
        \item{
            \Keywords{$\Inverse{R}[Y]$ $=$ $\domain{R}$}.
        }
        \item{
            If $a$ $\not \in$ $\domain{R}$,
            \Keywords{$R[\Set{a}]$ $=$ $\emptyset$}.
        }
        \item{
            If $b$ $\not \in$ $\range{R}$,
            \Keywords{$\Inverse{R}[\Set{b}]$ $=$ $\emptyset$}.
        }
        \item{
            \Keywords{$\domain{R}$ $=$ $\range{\Inverse{R}}$}.
        }
        \item{
            \Keywords{$\range{R}$ $=$ $\domain{\Inverse{R}}$}.
        }
        \item{
            \Keywords{$\Inverse{(\Inverse{R})}$ $=$ $R$}.
        }
        \item{
            \Keywords{$\Inverse{R} \composition R$ $\supseteq$ ${\identityrelation}_{\domain{R}}$}.
        }
        \item{
            \Keywords{$R \composition \Inverse{R}$ $\supseteq$ ${\identityrelation}_{\range{R}}$}.
        }
    }
}{
    Obviously.
}
\Note{
    If there exists one direction relationship from the domain or the range of the relation to another one, the relationship is in the relation of the composition of the relation and its reversion.
}

\Property{}{
    \List{enumerate}{
        \item{
            \Keywords{$A \cartesianproduct B$ $=$ $\emptyset$} if and only if \Keywords{$A$ $=$ $\emptyset$} or \Keywords{$B$ $=$ $\emptyset$}.
        }
        \item{
            \Keywords{$(A_{1} \cup A_{2}) \cartesianproduct B$ $=$ $(A_{1} \cartesianproduct B) \cup (A_{2} \cartesianproduct B)$}.
        }
        \item{
            \Keywords{$(A_{1} \cap A_{2}) \cartesianproduct B$ $=$ $(A_{1} \cartesianproduct B) \cap (A_{2} \cartesianproduct B)$}.
        }
        \item{
            \Keywords{$(A_{1} - A_{2}) \cartesianproduct B$ $=$ $(A_{1} \cartesianproduct B) - (A_{2} \cartesianproduct B)$}.
        }
        \item{
            \Keywords{$(A_{1} \symmetricdifference A_{2}) \cartesianproduct B$ $=$ $(A_{1} \cartesianproduct B) \symmetricdifference (A_{2} \cartesianproduct B)$}.
        }
    }
}{
    Obviously.
}