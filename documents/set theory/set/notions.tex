Unter einer Menge verstehen wir jede Zusammenfassung M von bestimmten wohlunterschiedenen Objekten in unserer Anschauung oder unseres Denkens (welche die Elemente von M genannt werden) zu einem ganzen.
\Definition{
	Set,
	Element (Member),
	belong to
}{
	A set is a collection of objects of our intuition.
	The objects are elements (members) of the set.
	We say that they belong to that set.
}

Property defines a set.
\Definition{
	Property
}{
	The notion of thinking of objects as being together.
}
\Definition{
	property of,
	depends on,
	Parameter
}{
	A proposition is a property of $X$, $Y$, {\dots}$Z$ if it holds or does not hold depending on sets (or called parameters) $X$, $Y$, {\dots}$Z$.
}
\Note{
	A variety of objects are bound by some common property, and form a set of objects having that property.
}
\Note{
	By merely defining a set, we do not prove its existence.
	There are properties which do not define sets.
}
\Definition{
	Statement
}{
	Properties which have no parameters.
}
\Note{
	Either true or false.
}
\Note{
	All mathematical theorems are (true) statements.
}
\Instance{
	Membership,
	belongs to
}{
	"{\dots} is an element (member) of {\dots},",
	"{\dots} belongs to {\dots}".
	Denote by $\in$.
}
\Note{
	Basic set-theoretic property.
}
\Definition{
	Proposition
}{
	Argument.	
}

Unspecified sets.
\Definition{
	Variable
}{
	Unspecified, arbitrary sets.
}
\Definition{
	same set as,
	identical with,
	equal to
}{
	$X = Y$ if $X$ is the same set as (is identical with, is equal to) $Y$.
}

All other set-theoretic properties can be stated in terms of membership with the help of logical means; identity, logical connectives, and quantifiers.
\Definition{
	Logical Connectives
}{
	Expressions like "not \dots ," " \dots and \dots. ," "if \dots , then \dots ," and " \dots if and only if \dots."
}
\Definition{
	Quantifiers
}{
	"for all" ("for every") and "there is" ("there exists").
}

\Definition{
	System of Sets,
	Collection of Sets
}{
	Elements of the set are sets.
}