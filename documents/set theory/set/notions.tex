\Definition{
	Set, Element (Member)
}{
	A set is a collection (group, or conglomerate) into a whole of definite, distinct objects of our intuition or our thought.
	The objects from which a given set is composed are called elements or members of the set. We also say that they belong to that set.
}
\Note{
	The human mind possesses the ability to abstract, to think of a variety of different objects as being bound together by some common property, and thus to form a set of objects having that property.
}
\Note{
	By merely defining a set we do not prove its existence. There are properties which do not define sets.
}

\Definition{
	Variable
}{
	Unspecified, arbitrary sets.
}
\Definition{
	identical with, equal to
}{
	We use the identity sign "$=$" to express that two variables denote the same set.
	$X = Y$ if $X$ is the same set as $Y$.
}

\Definition{
	System of Sets, Collection of Sets
}{
	Elements of the set are sets.
}

\Definition{
	Property
}{
	The property is the ability to think of these objects (as being) together.
}
\Instance{
	Membership
}{
	"{\dots} is an element (member) of {\dots},",
	"{\dots} belongs to {\dots}."
}
\Note{
	Denote by $\in$.
}
\Note{
	All other set-theoretic properties can be stated in terms of membership with the help of logical means; identity, logical connectives, and quantifiers.
}
\Definition{
	property of, Parameter
}{
	A proposition is a property of $X$, $Y$, {\dots}$Z$ if it holds or does not hold depending on sets (or called parameters) denoted by $X$, $Y$, {\dots}$Z$.
}

\Definition{
	Subset, included in
}{
	$A$ is a subset of (included in) $B$ if, for every $x$, $x \in A$ implies $x \in B$.
}
\Note{
	Denoted by $A \subseteq B$.
}
\Definition{
	Inclusion
}{
	The property $\subseteq$ is called inclusion.
}
\Theorem{}{
	$A \subseteq A$.\newline
	If $A \subseteq B$ and $B \subseteq A$, then $A = B$.\newline
	If $A \subseteq B$ and $B \subseteq C$, then $A \subseteq C$.
}{
	Obviously.
}
\Definition{
	Proper Subset, properly contained in
}{
	If $A \subseteq B$ and $A \subset B$, we say that $A$ is a proper subset of $B$ ($A$ is properly contained in $B$) and write $A \subset B$.
}
\Definition{
	Intersection
}{
	The intersection of $A$ and $B$, $A \cap B$, is the set of all $x$ which belong to both $A$ and $B$.
}
\Theorem{}{
	$\cap emptyset$ would have to be a set of all sets.
}{
	Obviously.
}
\Definition{
	Disjoint
}{
	$A$ and $B$ are disjoint if $A \cap B = emptyset$.
}
\Definition{
	Mutually Disjoint
}{
	$S$ is a system of mutually disjoint sets if $A \cap B = emptyset$ for all $A, B \in S$ such that $A \neq B$.
}
\Definition{
	Union
}{
	The union of $A$ and $B$, $A \cup B$, is the set of all $x$ which belong in either $A$ or $B$ (or both).
}
\Definition{
	Difference
}{
	The difference of $A$ and $B$, $A - B$, is the set of all $x \in A$ which do not belong to $B$.
}
\Definition{
	Symmetric Difference
}{
	The symmetric difference of $A$ and $B$, $A \triangle B$, is defined by $A \triangle B = (A-B) \cup (B-A)$.
}

\Definition{
	Statement
}{
	Properties which have no parameters.
}
\Note{
	Either true or false.
}
\Note{
	All mathematical theorems are (true) statements.
}
\Definition{
	Proposition
}{
	Argument.	
}