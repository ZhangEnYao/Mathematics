\Axiom{
	Existence
}{
	There exists a set which has no elements.
}
\Axiom{
	Extensionality
}{
	If every element of $X$ is an element of $Y$ and every element of $Y$ is an element of $X$, then $X = Y$.
}
\Lemma{}{
	There exists only one set with no elements.
}{
	Obviously.
}
\Definition{
	Empty, Vacuous Set
}{
	The (unique) set with no elements is called the empty, vacuous set and is denoted $\emptyset$.
}
\Note{
	Occasionally refer to $\emptyset$ as the constant.
}
\Axiom{
	Comprehension
}{
	Let $P(x)$ be a property of $x$. For any set $A$, there is a set $B$ such that  $x \in B$ if and only if $x \in A$ and $P(x)$.
}
\Note{
	For any sets $p$, {\dots}, $q$ and any $A$, there is a set $B$ (depending on $p$, {\dots}, $q$ and, of course, on $A$) consisting exactly of all those $x \in A$ for which $P(x, p, {\dots}, q)$.
}
\Lemma{}{
	For every $A$, there is only one set $B$ such that $x \in B$ if and only if $x \in A$ and $P(x)$.
}{
	Obviously.
}
\Definition{
	$\Set{x \in A \mid P(x)}$
}{
	The set of all $x \in A$ with the property $P(x)$.
}
\Note{
	If there is a set $A$ such that, for all $x$, $P(x)$ implies $x \in A$, then $\Set{x \in A \mid P(x)}$ exists, and, moreover, does not depend on $A$.
	That means that if $A'$ is another set such that for all $x$, $P(x)$ implies $x \in A'$, then $\Set{x \in A' \mid P(x)} = \Set{x \in A \mid P(x)}$.
}
\Definition{
	$\Set{x \mid P(x)}$
}{
	$\Set{x \mid P(x)}$ to be the set $\Set{x \in A \mid P(x)}$, where $A$ is any set for which $P(x)$ implies $x \in A$. 
}
\Note{
	$\Set{x \mid P(x)}$ is the set of all $x$ with the property $P(x)$.
}
\Note{
	This notation can be used only after it has been proved that some $A$ contains all $x$ with the property $P$.
}
\Property{}{
	$\Set{x \in \emptyset \mid P(x)} = \emptyset$.
}{
	Obviously.
}
\Lemma{}{
	If $P$ and $Q$ are sets, then there is a set $R$ such that $x \in R$ if and only if $x \in P$ and $x \in Q$.
}{
	Obviously.
}
\Definition{
	Intersection, Operation
}{
	We can introduce a name, say $X \cap Y$, and call $X \cap Y$ the intersection of $X$ and $Y$.
	$\cap$ as the operation.
}
\Axiom{
	Pair
}{
	For any $A$ and $B$, there is a set $C$ such that $x \in C$ if and only if $x = A$ or $x = B$.
}
\Note{
	$A \in C$ and $B \in C$, and there are no other elements of $C$.
}
\Property{}{
	The set $C$ is unique.
}{
	Obviously.
}
\Definition{
	Unordered Pair
}{
	Unordered pair of $A$ and $B$ as the set having exactly $A$ and $B$ as its elements.
}
\Axiom{
	Union
}{
	For any set $S$, there exists a set $U$ such that $x \in U$ if and only if $x \in A$ for some $A \in S$.
}
\Property{}{
	Set $U$ is unique.
}{
	Obviously.
}
\Definition{
	Union
}{
	$U$ is the union of $S$.
}
\Note{
	Denoted by $\bigcup S$.
}
\Axiom{
	Power Set
}{
	For any set $S$, there exists a set $P$ such that $X \in P$ if and only if $X \subseteq S$.
}
\Property{}{
	The set $P$ is uniquely determined.
}{
	Obviously.
}
\Note{
	We call the set of all subsets of $S$ the power set of $S$ and denote it by $\mathcal{P}(S)$.
}