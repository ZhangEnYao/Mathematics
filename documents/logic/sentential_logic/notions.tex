\Definition{
	Metalanguage
}{
	The language in which we talk about another language.
}
\Definition{
	Object Language
}{
	The language we are studying.
}
\Remark{
	The natural language is our metalanguage, and sentential logic is our object language.
}
\Definition{
	Compound Sentence
}{
	A sentence containing at least one sentence connective.
}
\Definition{
	Component Sentence
}{
	The sentence or sentences operated on by a truth-functional operator.
}
\Definition{
	Sentence Connective
}{
	A term or phrase used to make a larger sentence from two smaller ones.
	\ExempliGratia{and, or, if \dots then, if and only if, not}
}
\Definition{
	Atomic(Simple) Sentence, Compound Sentence
}{
	\Notion{Atomic} sentence is a sentence that contains no sentence connectives;
	\Notion{Compound} sentence is a sentence containing at least one sentence connective.
}
\Remark{
	Compound Sentences are sentences built from shorter sentences by means of sentence connectives.
}
\Note{
	\Notion{Sentence logic} can be developed without considering the interior structure of atomic sentences.
	\Notion{Symbolic logic} is the modern logic that includes sentential logic and predicate logic.
}
\Definition{
	Truth-Value
}{
	There are two truth-values—namely, true and false.
}
\Remark{
	Every sentence in sentential logic has a definite truth-value, and the truth-value of every compound sentence is a function of the truth-value of its component (or atomic) sentences.
}
\Definition{
	Truth Condition
}{
	The conditions under which a statement is true are called its truth conditions.
}
\Definition{
	Truth-Function
}{
	A function that takes one or more truth-values as its input and returns a single truth-value as its output.
}
\Definition{
	Truth-Functional Operator
}{
	An operator is truth-functional if the truth-values of the sentences formed by its use are determined by the truth-values of the sentences it connects.
	Similarly, sentence forms constructed by means of truth-functional sentence connectives are such that the truth-values of their substitution instances are determined by the truth-values of their component sentences.
}
\Remark{
	Our system of logic has five truth-functional operators.
	One operator, not, takes only one input; the other four, and, or, if {\dots} then, and if and only if, take two.
}
\Remark{
	The meaning of a logical operator is given by its truth conditions.
}
\Definition{
	Truth Table
}{
	A table giving the truth-values of all possible substitution instances of a given sentence form, in terms of the possible truth-values of the component sentences of these substitution instances.
}
\Remark{
	A statement does not have a truth table;
	a statement has a line in a truth table, depending on the truth-value of its component statements.
	Truth tables are not given for statements but only for statement forms.
}
\Definition{
	Logically Equivalent
}{
	A sentence $p$ is logically equivalent to a sentence $q$ if and only if $p$ $\equiv$ $q$ is a tautology.
}
\Definition{
	Scope
}{
	The scope of an operator is the component sentence or sentences that the operator operates on.
	The negation operator operates on a single component sentence.
	All the other operators operate on two component sentences.
}
\Remark{
	\Notion{Parentheses} are used to indicate the scope of each logical operator in any sentence.
}
\Remark{
	The scope of the negation operator is always the shortest complete sentence that follows it.
}
\Definition{
	Main Connective
}{
	The sentence connective that has the greatest scope.
	The main connective of a sentence is the truth-functional connective whose scope encompasses the entire remainder of the sentence.
}
\Definition{
	Well-Formed
}{
	A well-formed statement is one constructed according to the formation rules of a language.
	Sometimes such a statement is called a well-formed formula, abbreviation wff.
}
\Remark{
	No sentence is legitimate--that is, well-formed--unless it is clear which operator is the main operator for the sentence and which operators have which component sentences within their scope. 
}
\Definition{
	Negation
}{
	A sentence whose main connective is “not,” “no,” or a similar term.
}
\Definition{
	Negation Operator
}{
	Let $T$ and $F$ be truth-value true and false seperately.
	The negation operator is defined as the set $\{(T, F), (F, T)\}$.
}
\Note{
	It operates only on individual sentences.
}
\Definition{
	Conjunction
}{
	In sentential logic, a compound sentence (or sentence form) whose main connective is conjunction connective.
	\Notion{conjunction connective} is “and,” “and on the contrary,” "also," “although,” “both \dots and,” “but,” “despite (in spite of) the fact that,” “however,” “on the other hand,” “still,” "while,” “yet,” or a similar term.
	\Notion{Conjunct} is one of the sentences joined together by conjunction connective.
}
\Definition{
	Conjunction Operator
}{
	Let $T$ and $F$ be truth-value true and false seperately.
	The conjunction operator is defined as the set $\{((T, T), T), ((T, F), F), ((F, T), F), ((F, F), F)\}$.
}
\Proposition{
}{
	The conjunction operator is commutative.
}{
	Let $P$, $Q$ be statement.\\
	\LogicProof{
		\LogicProofStatement
		{	\ConditionalProofArray{ImplicationPremise}{1.}
		}{	$P \cdot Q$
		}{	Assumed Premise
		}
		\LogicProofStatement
		{	2.
		}{	$P$
		}{	1
		}
		\LogicProofStatement
		{	3.
		}{	$Q \cdot P$
		}{	2
		}
		\LogicProofStatement
		{	4.
		}{	$(P \cdot Q) \implies (Q \cdot P)$
		}{	\ConditionalProofArray{ImplicationConclusion}{1, 3, Conditional Proof}
		}
		\LogicProofStatement
		{	\ConditionalProofArray{ConversePremise}{5.}
		}{	$Q \cdot P$
		}{	Assumed Premise
		}
		\LogicProofStatement
		{	6.
		}{	$P$
		}{	5
		}
		\LogicProofStatement
		{	7.
		}{	$P \cdot Q$
		}{	6
		}
		\LogicProofStatement
		{	8.
		}{	$(Q \cdot P) \implies (P \cdot Q)$
		}{	\ConditionalProofArray{ConverseConclusion}{5, 7, Conditional Proof}
		}
	}\\
	\ConditionalProofArrayImplement{ImplicationPremise}{ImplicationConclusion}{2.3}
	\ConditionalProofArrayImplement{ConversePremise}{ConverseConclusion}{2.3}
}
\Remark{
	$p$ and $q$ have the same truth-values in every line of the truth table.
}
\Algorithm{
	Test of a Proposed Symbolization
}{
	If it is possible for your proposed symbolization and the sentence in natural language would have different truth-values under the same circumstances, the proposed symbolization is not an adequate translation.
}
\Algorithm{
	Test of the Truth-Functional Property of a Connective
}{
	If a connective is truth-functional, you will never be able to describe situations where in one case the sentence is true and in the other the sentence is false and yet the truth-value of the component sentence or sentences is the same in both cases.
}
