\Definition{
	Sentence
}{
	Undefined object.
}
\Definition{
	Statement
}{
	The use of a sentence that has a definite, fixed truth-value.
}
\Remark{
	The sentences in an argument must express statements—that is, say something that is either true or false.
}
\Definition{
	Sentence Constant
}{
	A symbol abbreviating an sentence in natural language, atomic or compound.
}
\Definition{
	Statement Variable
}{
	A symbol that represents no statement and thus has no truth-value but for which statements can be substituted.
}
\Remark{
	What does have a truth-value is a statement we substitute for it, 
	The truth-value of the statement substituted for the variable will thus vary according to which statement is substituted.
}
\Definition{
	Substitution
}{
	An operation performed by putting a statement in place of a variable.
}
\Remark{
	We obtain statements from statement forms by substitution.
}
\Assumption{
}{
	All statements have a fixed \Notion{truth-value}—that is, are either true or false.
}
\Remark{
	A sentence can be used to make different statements, depending on circumstances.
}
\Definition{
	Argument
}{
	A set of sentences, one of which, the conclusion, is claimed to be supported by the others, the premises.
	The \Notion{premises} are the reasons given in support of an argument’s conclusion.
	The \Notion{conclusion} is the statement in an argument that is argued for on the basis of the argument’s premises.
}
\Remark{
	Not just any group of sentences makes an argument.
}
\Definition{
	Argument Form
}{
	Informally, the logical structure of an argument.
	Formally, a group of sentence forms, all of whose substitution instances are arguments.
}
\Definition{
	(Deductively) Valid
}{
	An argument is \Notion{valid} if and only if it is not possible for all of its premises to be true and its conclusion false.
}
\Theorem{
}{
	If all premises of a valid argument are true, then its conclusion must be true. 
}{
	Let $P$ be all premises of a valid argument are true and $C$ be its conclusion must be true.\\
	\LogicProof{
		\LogicProofStatement
		{	1.
		}{	$P$
		}{	Premise
		}
		\LogicProofStatement
		{	2.
		}{	$\neg( {P} \land \neg{Q} )$
		}{	Definition: valid argument
		}
		\LogicProofStatement
		{	3.
		}{	$\neg{P} \lor {Q}$
		}{	2
		}
		\LogicProofStatement
		{	4.
		}{	$Q$
		}{	1, 3
		}
	}\\
}
\Corollary{
}{
	The conclusion of a valid argument follows logically from the premises.
}{
	Obviously.
}
\Note{
	The truth of the premises of a valid argument guarantees the truth of its conclusion.
}
\Remark{
	If all you know about an argument is that it is valid, that alone tells you nothing about whether the premises are true or the conclusion is true.
	From the mere fact that an argument is invalid you can draw no conclusion whatsoever about the truth or falsity of the premises or the conclusion.
}
\Definition{
	Inductive Argument
}{
	An argument that is not valid but whose premises provide some measure of support for its conclusion.
}
\Remark{
	It is possible for the premises of a strong inductive argument to be true and yet the conclusion be false.
}
\Note{
	An inductively strong argument does not guarantee that if its premises are true, then its conclusion also will be true, it does make its conclusion more probable.
}
\Definition{
	Sound Argument
}{
	An argument that is valid and has all true premises.
}
\Theorem{
}{
	It is not possible that a deductively valid argument be sound, yet have a false conclusion.
}{
	Let $S$ be a deductively valid argument is sound $P$ be all premises are true and $C$ be the conclusion is true.\\
	\LogicProof{
		\LogicProofStatement
		{	1.
		}{	${S} \land \neg{C}$
		}{	Assumed Premise
		}
		\LogicProofStatement
		{	2.
		}{	${P} \implies {C} \land {P}$
		}{	Definition: Sound
		}
		\LogicProofStatement
		{	3.
		}{	$C$
		}{	2
		}
		\LogicProofStatement
		{	4.
		}{	$( {S} \land \neg{C} ) \land {C}$
		}{	1, 3
		}
		\LogicProofStatement
		{	5.
		}{	${S} \land ( \neg{C} \land {C} )$
		}{	4
		}
		\LogicProofStatement
		{	6.
		}{	$\neg( {S} \land \neg{C} )$
		}{	5, Indirect Proof
		}
	}\\
}
\Definition{
	Consistent
}{
	A set of statements is consistent if and only if it is possible for all the statements to be true.
}
\Theorem{
}{
	If an argument is valid, then it is not possible for its premises to be true and its conclusion false.
}{
	Let $V$ be an argument is valid, $P$ be all premises are true and $C$ be the conclusion is true.\\
	\LogicProof{
		\LogicProofStatement
		{	1.
		}{	${P} \implies {Q}$
		}{	Definition: Valid Argument
		}
		\LogicProofStatement
		{	2.
		}{	$\neg{P} \lor {Q}$
		}{	1
		}
		\LogicProofStatement
		{	3.
		}{	$\neg( {P} \land \neg{Q} )$
		}{	2
		}		
	}\\
}
\Corollary{
}{
	An argument is valid if and only if it is inconsistent to say that all its premises are true and its conclusion is false.
}{
	Obviously.
}