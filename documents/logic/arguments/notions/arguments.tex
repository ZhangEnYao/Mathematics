\Definition{
	Statement
}{
	The use of a sentence that has a definite, fixed truth-value.
}
\Remark{
	The sentences in an argument must express statements—that is, say something that is either true or false.
}
\Definition{
	Sentence Constant
}{
	A symbol abbreviating an sentence in natural language, atomic or compound.
}
\Definition{
	Statement Variable
}{
	A symbol that represents no statement and thus has no truth-value but for which statements can be substituted.
}
\Remark{
	What does have a truth-value is a statement we substitute for it, 
	The truth-value of the statement substituted for the variable will thus vary according to which statement is substituted.
}
\Definition{
	Substitution
}{
	An operation performed by putting a statement in place of a variable.
}
\Remark{
	We obtain statements from statement forms by substitution.
}
\Assumption{
}{
	All statements have a fixed \Domain{truth-value}—that is, are either true or false.
}
\Remark{
	A sentence can be used to make different statements, depending on circumstances.
}
\Definition{
	Argument
}{
	A set of sentences, one of which, the conclusion, is claimed to be supported by the others, the premises.
	The \Domain{premises} are the reasons given in support of an argument’s conclusion.
	The \Domain{conclusion} is the statement in an argument that is argued for on the basis of the argument’s premises.
}
\Remark{
	Not just any group of sentences makes an argument.
}
\Definition{
	Argument Form
}{
	Informally, the logical structure of an argument.
	Formally, a group of sentence forms, all of whose substitution instances are arguments.
}