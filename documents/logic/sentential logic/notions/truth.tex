\Definition{
	Truth-Value
}{
	There are two truth-values—namely, true and false.
}
\Remark{
	Every sentence in sentential logic has a definite truth-value, and the truth-value of every compound sentence is a function of the truth-value of its component (or atomic) sentences.
}
\Definition{
	Truth Condition
}{
	The conditions under which a statement is true are called its truth conditions.
}
\Definition{
	Truth-Function
}{
	A function that takes one or more truth-values as its input and returns a single truth-value as its output.
}
\Definition{
	Truth-Functional Operator
}{
	An operator is truth-functional if the truth-values of the sentences formed by its use are determined by the truth-values of the sentences it connects.
	Similarly, sentence forms constructed by means of truth-functional sentence connectives are such that the truth-values of their substitution instances are determined by the truth-values of their component sentences.
}
\Remark{
	Our system of logic has five truth-functional operators.
	One operator, not, takes only one input; the other four, and, or, if {\dots} then, and if and only if, take two.
}
\Remark{
	The meaning of a logical operator is given by its truth conditions.
}
\Definition{
	Truth Table
}{
	A table giving the truth-values of all possible substitution instances of a given sentence form, in terms of the possible truth-values of the component sentences of these substitution instances.
}
\Remark{
	A statement does not have a truth table;
	a statement has a line in a truth table, depending on the truth-value of its component statements.
	Truth tables are not given for statements but only for statement forms.
}