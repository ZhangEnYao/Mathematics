\Definition{
	Scope
}{
	The scope of an operator is the component sentence or sentences that the operator operates on.
	The negation operator operates on a single component sentence.
	All the other operators operate on two component sentences.
}
\Remark{
	\Domain{Parentheses} are used to indicate the scope of each logical operator in any sentence.
}
\Remark{
	The scope of the negation operator is always the shortest complete sentence that follows it.
}
\Definition{
	Main Connective
}{
	The sentence connective that has the greatest scope.
	The main connective of a sentence is the truth-functional connective whose scope encompasses the entire remainder of the sentence.
}
\Definition{
	Well-Formed
}{
	A well-formed statement is one constructed according to the formation rules of a language.
	Sometimes such a statement is called a well-formed formula, abbreviation wff.
}
\Remark{
	No sentence is legitimate--that is, well-formed--unless it is clear which operator is the main operator for the sentence and which operators have which component sentences within their scope. 
}
\Definition{
	Negation
}{
	A sentence whose main connective is “not,” “no,” or a similar term.
}
\Definition{
	Negation Operator
}{
	Let $T$ and $F$ be truth-value true and false seperately.
	The negation operator is a truth-functional connective defined as the set \Set{$(T, F)$, $(F, T)$}.
}
\Note{
	It operates only on individual sentences.
}
\Definition{
	Conjunction
}{
	In sentential logic, a compound sentence (or sentence form) whose main connective is conjunction connective.
	\Domain{Conjunction connective} is “and,” “and on the contrary,” "also," “although,” “both \dots and,” “but,” “despite (in spite of) the fact that,” “however,” “on the other hand,” “still,” "while,” “yet,” or a similar term.
	\Domain{Conjunct} is one of the sentences joined together by conjunction connective.
}
\Definition{
	Conjunction Operator
}{
	Let $T$ and $F$ be truth-value true and false seperately.
	The conjunction operator is a truth-functional connective defined as the set \Set{$((T, T), T)$, $((T, F), F)$, $((F, T), F)$, $((F, F), F)$}.
}
\Proposition{
}{
	The conjunction operator is commutative.
}{
	Let $P$, $Q$ be statements.\\
	\LogicProof{
		\LogicProofStatement
		{	\ConditionalProofArrow{ImplicationPremise}{1.}
		}{	$P \cdot Q$
		}{	Assumed Premise
		}
		\LogicProofStatement
		{	2.
		}{	$P$
		}{	1
		}
		\LogicProofStatement
		{	3.
		}{	$Q \cdot P$
		}{	2
		}
		\LogicProofStatement
		{	4.
		}{	$(P \cdot Q) \implies (Q \cdot P)$
		}{	\ConditionalProofArrow{ImplicationConclusion}{1, 3, Conditional Proof}
		}
		\LogicProofStatement
		{	\ConditionalProofArrow{ConverseImplicationPremise}{5.}
		}{	$Q \cdot P$
		}{	Assumed Premise
		}
		\LogicProofStatement
		{	6.
		}{	$P$
		}{	5
		}
		\LogicProofStatement
		{	7.
		}{	$P \cdot Q$
		}{	6
		}
		\LogicProofStatement
		{	8.
		}{	$(Q \cdot P) \implies (P \cdot Q)$
		}{	\ConditionalProofArrow{ConverseImplicationConclusion}{5, 7, Conditional Proof}
		}
		\LogicProofStatement
		{	9.
		}{	$(Q \cdot P) \iff (P \cdot Q)$
		}{	4, 8
		}
	}
	\ConditionalProofArrowImplement{ImplicationPremise}{ImplicationConclusion}{2.3em}
	\ConditionalProofArrowImplement{ConverseImplicationPremise}{ConverseImplicationConclusion}{2.3em}
}
\Remark{
	$p$ and $q$ have the same truth-values in every line of the truth table.
}
\Definition{
	Disjunctions
}{
	A compound sentence whose main connective is an “or.”
	\Domain{Disjunctions connective} is “or,” “either . . . or,” or a similar term.
	\Domain{Disjunct} is either of the component sentences in a disjunction.
}
\Remark{
	There are two different senses of the connective “or” in common use.
}
\Definition{
	Disjunction Operator
}{
	Let $T$ and $F$ be truth-value true and false seperately.
	The disjunction operator is a truth-functional connective defined as the set \Set{$((T, T), T)$, $((T, F), T)$, $((F, T), T)$, $((F, F), F)$}.
}
\Definition{
	Exclusive “Or”
}{
	One and only one of the two disjuncts is true, not both.
	Or, at least one disjunct is false.
}
\Definition{
	Exclusive Disjunction
}{
	A compound sentence whose main connective is an exclusive "or."
}
\Definition{
	Inclusive (Nonexclusive) “Or”, or And/Or
}{
	At least one of the two disjuncts is true, but leaves open the possibility that both disjuncts are true.
	Or, at least one of its disjuncts is true.
}
\Remark{
	A sentence whose major connective is an exclusive “or” asserts more than it would if the “or” were inclusive.
	The inclusive “or” is only part of the meaning of the exclusive “or.”
}
\Definition{
	Conditional, or Hypothetical
}{
	A compound sentence that expresses an “If \dots then” relationship between its component sentences.
	The sentence follows "if" in nan "if \dots then" statements is called \Domain{antecedent}, and the sentence follows the “then” is called \Domain{consequent}.
	The consequent is \Domain{necessary condition} for the antecedent; the antecedent expresses a sufficient condition.
}
\Remark{
	Let $P$, $Q$ be statements.
	$P$ only if $Q$ is logically equivalent to $\neg{P} \implies \neg{Q}$.
}
\Note{
	A material conditional is essentially a statement that the truth of the antecedent is sufficient for the truth of the consequent and that the truth of the consequent is necessary for the truth of the antecedent.
	The consequent alone may not be sufficient to guarantee the consequent being true.
}
\Definition{
	Material Implication
}{
	Let $T$ and $F$ be truth-value true and false seperately.
	The conditional operator is a truth-functional connective (or truth-function) defined as the set \Set{$((T, T), T)$, $((T, F), F)$, $((F, T), T)$, $((F, F), F)$}.
}
\Definition{
	Material Conditional
}{
	A compound statement whose main connective is material implication.
	A compound statement expresses an “If . . . then” relationship between its component sentences.
}
\Remark{
	Conditional sentences differ with respect to the kind of connection they express between anteced- ent and consequent.
	The connection between antecedent and consequent is logical, if the antecedent is true implies the consequent must be;
	the connection is causal if the conditions stated in the antecedent cause the conditions stated in the consequent.
}
\Note{
	The general form of a conditional sentence is "if (antecedent), then (consequent)."
}
\Definition{
	Materially Equivalent
}{
	Let $P$, $Q$ be statements.
	$P$, $Q$ are said to be \Domain{materially equivalent} (or $P$ if and only if $Q$) when $P$, $Q$ have the same truth-value.
}
\Definition{
	Logically Equivalent
}{
	A sentence $p$ is logically equivalent to a sentence $q$ if and only if $p \equiv q$ is a tautology.
	That is, two sentences have exactly the same truth conditions.
}
\Definition{
	Material Equivalence Operator
}{
	Let $T$ and $F$ be truth-value true and false seperately.
	The material equivalence operator is a truth-functional connective (or truth-function) defined as the set \Set{$((T, T), T)$, $((T, F), F)$, $((F, T), F)$, $((F, F), T)$}.
}
\Definition{
	Material Equivalences (or Material Biconditionals, Biconditionals)
}{
	A compound statement whose main sentence connective is material equivalence operator.
	A compound statement expresses an “if and only if” relationship between two component sentences.
}
\Remark{
	Material Equivalences are themselves equivalent to two-directional material conditionals.
}
\Algorithm{
	Test of the Truth-Functional Property of a Connective
}{
	If a connective is truth-functional, you will never be able to describe situations where in one case the sentence is true and in the other the sentence is false and yet the truth-value of the component sentence or sentences is the same in both cases.
}
\Notion{
	Many sentences contain more than one logical operator.
}{
	\ExempliGratia{Not \dots Both, Neither \dots Nor, Unless.}
	Let $P$, $Q$ be statements.
	Not both $P$ and $Q$ is symbolized $\neg$($P \land Q$);
	Neither $P$ nor $Q$ is symbolized $\neg{P} \land \neg{Q}$;
	$P$ unless $Q$ is symbolized $\neg{Q} \implies P$ (or $P \lor Q$).
}
\Claim{
}{
	$\neg{Q}$ $\implies$ $P$ is logically equivalent to $P \lor Q$.
}{
	Let $P$, $Q$ be statements.\\
	\LogicProof{
		\LogicProofStatement
		{	\ConditionalProofArrow{ImplicationPremise}{1.}
		}{	$\neg{Q} \implies P$
		}{	Assumed Premise
		}
		\LogicProofStatement
		{	2.
		}{	$\neg{\neg{Q}} \lor P$
		}{	1
		}
		\LogicProofStatement
		{	3.
		}{	$Q \lor P$
		}{	2
		}
		\LogicProofStatement
		{	4.
		}{	$P \lor Q$
		}{	3
		}
		\LogicProofStatement
		{	5.
		}{	$(\neg{Q} \implies P) \implies (P \lor Q)$
		}{	\ConditionalProofArrow{ImplicationConclusion}{1, 4, Conditional Proof}
		}
		\LogicProofStatement
		{	\ConditionalProofArrow{ConverseImplicationPremise}{6.}
		}{	$P \lor Q$
		}{	Assumed Premise
		}
		\LogicProofStatement
		{	7.
		}{	$Q \lor P$
		}{	6.
		}
		\LogicProofStatement
		{	8.
		}{	$\neg{\neg{Q}} \lor P$
		}{	7.
		}
		\LogicProofStatement
		{	9.
		}{	$\neg{Q} \implies P$
		}{	8.
		}
		\LogicProofStatement
		{	10.
		}{	$(P \lor Q) \implies (\neg{Q} \implies P)$
		}{	\ConditionalProofArrow{ConverseImplicationConclusion}{6, 9, Conditional Proof}
		}
		\LogicProofStatement
		{	11.
		}{	$(P \lor Q) \iff (\neg{Q} \implies P)$
		}{	5, 10
		}
	}
	\ConditionalProofArrowImplement{ImplicationPremise}{ImplicationConclusion}{2.3em}
	\ConditionalProofArrowImplement{ConverseImplicationPremise}{ConverseImplicationConclusion}{2.3em}
}