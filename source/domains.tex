% Structure
\NewDocumentCommand{\Notion}{ m }{%
	\textbf{#1}%
}
% Instance
\NewDocumentCommand{\ExempliGratia}{ m }{%
	e.g., {#1}%
}
\NewDocumentCommand{\Instance}{ m m }{%
	\begin{instance}\textbf{\emph{#1}}\\
		\normalfont{#2}
	\end{instance}
}
% Proof
\NewDocumentCommand{\LogicProof}{ m }{%
	\begin{tabular}{ l l l }
		#1
	\end{tabular}
}
\NewDocumentCommand{\LogicProofStatement}{ m m m }{%
	{#1} & {#2} & {#3}\\%
}
\newcommand{\ConditionalProofArray}[3][]{%
  	\tikz[remember picture, baseline=({#2}.base)]
    	\node[minimum size=0pt,inner sep=0pt, {#1}]({#2}){{#3}};
}
\NewDocumentCommand{\ConditionalProofArrayImplement}{ m m m }{%
    \begin{tikzpicture}[remember picture, overlay]
        \draw[stealth-] ({#1}) -- ++(-{#3}em,0) |- ({#2}.north east);
    \end{tikzpicture}
}
% Definition, Theorem, Remark, ...
\NewDocumentCommand{\Axiom}{ m m }{%
	\begin{axiom}\textbf{\emph{#1}}\\
		\normalfont{#2}
	\end{axiom}
}
\NewDocumentCommand{\Definition}{ m m }{%
	\begin{definition}\textbf{\emph{#1}}\\
		\normalfont{#2}
	\end{definition}
}
\NewDocumentCommand{\Assumption}{ m m }{%
	\begin{assumption}\textbf{\emph{#1}}\\
		\normalfont{#2}
	\end{assumption}
}
\NewDocumentCommand{\Algorithm}{ m m }{%
	\begin{algorithm}\textbf{\emph{#1}}\\
		\normalfont{#2}
	\end{algorithm}
}
\NewDocumentCommand{\Theorem}{ m m m }{%
	\begin{theorem}\textbf{\emph{#1}}\\
		\normalfont{#2}
	\end{theorem}
	\begin{proof}
		{#3}
	\end{proof}
}
\NewDocumentCommand{\Proposition}{ m m m }{%
	\begin{proposition}\textbf{\emph{#1}}\\
		\normalfont{#2}
	\end{proposition}
	\begin{proof}
		{#3}
	\end{proof}
}
\NewDocumentCommand{\Property}{ m m m }{%
	\begin{property}\textbf{\emph{#1}}\\
		\normalfont{#2}
	\end{property}
	\begin{proof}
		{#3}
	\end{proof}
}
\NewDocumentCommand{\Claim}{ m m m }{%
	\begin{claim}\textbf{\emph{#1}}\\
		\normalfont{#2}
	\end{claim}
	\begin{proof}
		{#3}
	\end{proof}
}
\NewDocumentCommand{\Lemma}{ m m m }{%
	\begin{lemma}\textbf{\emph{#1}}\\
		\normalfont{#2}
	\end{lemma}
	\begin{proof}
		{#3}
	\end{proof}
}
\NewDocumentCommand{\Corollary}{ m m m }{%
	\begin{corollary}\textbf{\emph{#1}}\\
		\normalfont{#2}
	\end{corollary}
	\begin{proof}
		{#3}
	\end{proof}
}
\NewDocumentCommand{\Remark}{ m }{%
	\begin{remark}
		\small{#1}
	\end{remark}
}
\NewDocumentCommand{\Note}{ m }{%
	\begin{note}
		\small{#1}
	\end{note}
}